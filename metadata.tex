% DO NOT EDIT - automatically generated from metadata.yaml

\def \codeURL{https://github.com/aleblois/BGdynamics2006}
\def \codeDOI{}
\def \dataURL{}
\def \dataDOI{}
\def \editorNAME{}
\def \editorORCID{}
\def \reviewerINAME{}
\def \reviewerIORCID{}
\def \reviewerIINAME{}
\def \reviewerIIORCID{}
\def \dateRECEIVED{17 April 2020}
\def \dateACCEPTED{}
\def \datePUBLISHED{}
\def \articleTITLE{[Re] Reproduction of the normal and pathological dynamics in the basal ganglia-thalamo-cortical network}
\def \articleTYPE{Reproduction}
\def \articleDOMAIN{}
\def \articleBIBLIOGRAPHY{bibliography.bib}
\def \articleYEAR{2020}
\def \reviewURL{}
\def \articleABSTRACT{In 2006, during my PhD, I published my first modelling paper showing that the normal and pathological dynamics of the basal ganglia-thalamo-cortical network could be understood from a theoretical standpoint relying on a simplified model of the circuit. I recently was told by a colleague about a challenge to reproduce modelling studies over 10-years old. I was quite confident I could reproduce the results of my 2006 paper, which was based on a combination of analytical calculation and computer simulations programmed in C with figures produced through Matlab custom code. All the code was still saved on my computer (although in a simple but relatively messy file-and-folder organisation), and I went through my personal archives and re-run ancient code until I could provide again the results described here. }
\def \replicationCITE{}
\def \replicationBIB{}
\def \replicationURL{}
\def \replicationDOI{}
\def \contactNAME{Arthur Leblois}
\def \contactEMAIL{arthur.leblois@u-bordeaux.fr}
\def \articleKEYWORDS{Reproduction, Neuroscience, Modelling, Parkinson’s disease, Oscillations}
\def \journalNAME{ReScience C}
\def \journalVOLUME{4}
\def \journalISSUE{1}
\def \articleNUMBER{}
\def \articleDOI{}
\def \authorsFULL{Arthur Leblois}
\def \authorsABBRV{A. Leblois}
\def \authorsSHORT{Leblois}
\title{\articleTITLE}
\date{}
\author[1,\orcid{0000-0002-9392-5939}]{Arthur Leblois}

\affil[1]{Institut des Maladies Neurodégénératives, UMR CNRS 5293, Neurocampus, Université de Bordeaux, 146 rue Léo Saignat, 33000 Bordeaux, France.}

